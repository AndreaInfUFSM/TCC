% !TEX TS-program = XeLaTeX+MakeIndex+BibTeX
% !TEX encoding = UTF-8 Unicode

\documentclass[12pt]{article}

\usepackage[utf8]{inputenc}
\usepackage[brazilian]{babel}

\usepackage{fontspec}
\setmainfont{Linux Libertine O}
\linespread{1.05}

%%% PAGE DIMENSIONS
\usepackage{geometry} % to change the page dimensions
\geometry{a4paper} % or letterpaper (US) or a5paper or....
% \geometry{margin=2in} % for example, change the margins to 2 inches all round
% \geometry{landscape} % set up the page for landscape

\usepackage{graphicx} % support the \includegraphics command and options

% \usepackage[parfill]{parskip} % Activate to begin paragraphs with an empty line rather than an indent

%%% PACKAGES
\usepackage{amsfonts}
\usepackage{color}
%\usepackage{booktabs} % for much better looking tables
%\usepackage{array} % for better arrays (eg matrices) in maths
%\usepackage{paralist} % very flexible & customisable lists (eg. enumerate/itemize, etc.)
\usepackage{verbatim} % adds environment for commenting out blocks of text & for better verbatim
\usepackage{microtype}
\usepackage[numbers]{natbib}
%\usepackage{subfig} % make it possible to include more than one captioned figure/table in a single float
% These packages are all incorporated in the memoir class to one degree or another...
\usepackage[hidelinks]{hyperref}

\usepackage{listings}


% For Computer Modern:
%\def\Cpp{{C\nolinebreak[4]\hspace{-.05em}\raisebox{.4ex}{\tiny\bf ++}}}
% For Linux Libertine G
\def\Cpp{{C\nolinebreak[4]\raisebox{.20ex}{\small\bf++}}}

\newcommand{\todo}[1]{\textsf{\color{red}#1}}

%%% END Article customizations

\title{AUTOMAÇÃO DE INTERFACES GRÁFICAS PARA MODELOS DE SIMULAÇÃO DE CULTURAS AGRÍCOLAS COM BASE EM TECNOLOGIA DE LINGUAGEM VISUAL}
\author{Romulo Pulcinelli Benedetti \\ \emph{Universidade Federal de Santa Maria}}
%\date{} % Activate to display a given date or no date (if empty), otherwise the current date is printed 

\begin{document}
	\maketitle
	
	\section{Identificação}
	
	\begin{description} \itemsep 0pt
		\item[Resumo:] ~\\
				 
	A automação de tarefas é uma forma bastante eficiente pela qual podemos reduzir custos e aumentar a produtividade e qualidade da atividade humana. A computação por si é uma ferramenta para atingir a automação de tarefas, com vários exemplos de softwares focados em automatizar tarefas específicas sob comando do usuário, sendo a automatização destes softwares um campo a parte. Observamos a aproximação destes softwares a abordagens mais naturais ao raciocínio humano, por meio de interfaces gráficas e contextualização dos elementos com base no mundo real, tornando estes softwares menos distantes do paradigma de interação do humano com a realidade. Desta forma este trabalho objetiva abordar a utilização de tecnologias voltadas a programação visual para melhorar a abordagem de automação de tarefas, com foco em softwares gráficos de modelagem matemática agrícola, assim inserindo a atividade de automatizar estes softwares gráficos, dentro do mesmo domínio de abstração que as atividades destes softwares ocorrem.
		 
		\item[Período de execução:] março de 2016 a julho de 2016
		\item[Unidades participantes:] ~\\ Curso de Ciência da Computação
		\item[Área de conhecimento:] Ciência da Computação
		\item[Linha de Pesquisa:] ?
		\item[Tipo de projeto:] Trabalho de Conclusão de Curso
		\item[Participantes:] ~\\ Profª Andrea Schwertner Charão -- Orientadora \\ Romulo Pulcinelli Benedetti -- Orientando
	\end{description}
	
	\section{Introdução}
	
	No desenvolvimento de software, áreas de conhecimento como engenharia de software e qualidade de software
investigam processos e normas, com o objetivo de reduzir a quantia de recursos necessários e garantir a qualidade de
software produzido. Um dos produtos destas áreas, envolvendo automação, foi o campo de conhecimento de testes de software.

Sendo que o software pode realizar diversas tarefas, e estas podem assumir diversos estados, qualquer abordagem
de teste de software, numa situação ideal deveria avaliar todas estas tarefas e seus estados para fornecer as melhores
informações possíveis sobre a qualidade do software, o que facilmente pode se tornar uma tarefa repetitiva e de longa
duração e na maioria dos casos nem sempre é possível, segundo \cite[pag. 10]{myers2011art}.

Embora nem sempre seja possível testar todos os estados possíveis, é possível realizar os testes em uma faixa
conhecida e finita de estados, nestes casos usar ferramentas como frameworks voltados a automatização destes testes
agiliza a tarefa de repetir os testes, como na abordagem de testes de regressão onde todos os testes devem ser
executados novamente a cada ciclo de desenvolvimento. Não o bastante oferece uma série de vantagens tais como precisão ao reduzir a necessidade da atenção humana durante o andamento da tarefa, agilizando atividades e melhorando o aproveitamento do tempo de trabalho humano.

Alguns destes frameworks, embora voltados a realização de testes, poderiam ser usados para automatizar programas gráficos, um exemplo é o Robot Framework. Entretanto, são ferramentas que, apesar de cobrirem de forma bastante detalhada a automação de testes, não são as mais adequadas a modelagem da automatização de softwares gráficos, dependem de um domínio de assuntos de diversos campos da área de Ciências da computação e em muitos casos, domínio da especificação e arquitetura do software a ser automatizado ou ainda alterações a nível de codificação no programa a ser automatizado.

Já outros casos particulares de automatização focada em tarefas de TI são os próprios terminais ou ainda shell e utilitários como o make e afins, ferramentas e linguagens voltadas para automatização do desenvolvimento e de tarefas administrativas, também inadequadas a automatização de programas gráficos, seja por serem bastante limitadas a um mundo formalmente textual, pela complexidade de sua sintaxe ou ainda pela abstração focadas em tarefas comuns apenas para profissionais de TI e softwares que oferecem interface com estes utilitários.

Temos também ferramentas destinadas a automação de software gráfico, como a linguagem script de automatização Autoit. Ainda assim ferramentas que exigem o domínio de um nível de abstração elevado, perceptivelmente diferente da abstração em que a atividade se da, em um ambiente gráfico, focado em facilidades visuais de interação.

A automatização de tarefas pode ser aproximada do usuário por meio de abordagens mais visuais e sintaxe mais contextualizada a modelagem de tarefas genéricas em interfaces gráficas, com o uso de linguagem visual para descrição das automações.

Modelos de simulação de culturas agrícolas vem sendo refinados para estar ao acesso de usuários da área a qual servem,  alguns exemplos de modelo em uso hoje em dia são SoySim, Simanihot e DSSAT (Decision Support System for Agrotechnology Transfer). Embora estejam se tornando significativamente mais fáceis de interagir por via gráfica, dependendo das atividades realizada e da forma como o modelo as realiza, trabalhar com estes programas pode se tornar uma atividade manual extremamente repetitiva e demorada. Tarefas como previsão de climas futuros é um dos exemplos mais notório desta situação.

A automatização por via de uma abordagem mais visual destes modelos permitiria a obtenção das vantagens aqui discutidas, sem deslocar o utilizador de sua área de domínio, a interface gráfica. 
	
	\section{Objetivos}
	
	\subsection{Objetivo Geral}
	
	Tornar possível a automação de tarefas em interfaces gráficas por meio da simplificação de uma abordagem visual com base na biblioteca de programação Visual Blockly dentro do contexto de modelos de simulação de culturas agrícolas.

	\subsection{Objetivos Específicos}
	
	\begin{itemize}
		\item Fornecer uma solução menos formal e textual, mais visual e contextualizada, de automatização;
		\item Automatizar tarefas computacionais em interfaces gráficas no campo de modelagem matemática agrícola;
		\item Auxiliar o campo de pesquisa e trabalho com modelos matemáticos de culturas agrícolas;
	\end{itemize}
	
	\section{Justificativa}	
	
	A automação de tarefas é hoje em dia um processo fundamental para a obtenção de resultados ágeis e de qualidade, tanto na produção de um produto ou no fornecimento de serviços, assim como na execução de tarefas pessoais. Representa também redução de custos, o que abre novas possibilidades permitindo trabalhos mais complexos e maiores chances de sucesso, entretanto ferramentas de automação na computação tem exigido um domínio de abordagens de abstração, que em geral, vão além da abstração com a qual, usuários que usam o computador como meio para outras áreas de conhecimento, que não a própria produção de softwares, estão acostumados. Uma abstração mais próxima ao nível em que estas tarefas ocorrem tornaria a automação um processo mais intuitivo.
	
	Uma destas áreas de conhecimento é a Fitotecnia, mais especificamente, envolvendo estudos do desenvolvimento da fenologia e produtividade de culturas que hoje utilizam modelos de simulação de culturas agrícolas que simulam estes processos, tendo evoluído nos últimos anos para programas com interfaces gráficas no intuito de alcançarem e beneficiarem mais pessoas, leigas em computação porem fluentes na área de conhecimento da ferramenta. Em alguns casos, as tarefas realizadas com estes simuladores envolvem interações repetitivas com o objetivo de obter uma grande faixa de amostragem de resultados, tarefas estas que se beneficiariam da automatização e estariam igualmente acessíveis ao domínio de seus usuários, caso esta automatização pudesse ser realizada dentro deste domínio de entendimento, um ambiente e uma metodologia visual.  
	
	\section{Revisão de Literatura}
	Na sequencia serão apresentados conceitos relativos aos conteúdos abordados nesse trabalho, descrevendo a automação, sua utilização dentro computação na área de TI e os resultados até então obtidos na automatização de tarefas mais genéricas, assim como linguagens de programação visual e a ferramenta com a qual este aspecto sera tratado.
	
	%Automação 
    %Automação na Computação     
    %Teste de Softwares
    %O que temos hoje     
      
    %- o que é necessário para compreender o trabalho
    %  - o que será usado no trabalho como recurso teórico
    %  - trabalhos relacionados não atendem o problema identificado
    %        < bem documentado >


	\subsection{Automação e suas abordagens dentro da computação}
	
	A automação é a execução de tarefas por meio de máquinas e computadores, antes executáveis apenas por humanos \cite{automationlevels}.	Segundo \citeauthor{automation2009} \cite[pág. 124]{automation2009}, a automação teve um impacto significante na economia e desenvolvimento tecnológico da sociedade, é um elemento chave para o alcance de produtos e serviços de alta qualidade e baixo custo. Dentre os vários aspectos da atividade humana. Uma das áreas impactadas pela automação foi a precisão em um ciclo auto alimentado, onde a automação melhora a precisão e por sua vez a precisão melhora a automação \cite{auto2008precision}.
	
	A automação da informação por meio de computadores, segundo \citeauthor{automation2009} \cite[pág. 3]{automation2009} é dentro da automação, um processo dos dias atuais, onde temos vendas automatizadas de passagens, conexões de chamadas em nossos telefones, realizadas de forma automática, dentre outras mudanças advindas da automatização, temos também a produção e manufatura por intermédio da robótica. No final das contas, o impacto da informática promoveu não só uma intensa automatização passiva, mas também, segundo \citeauthor{itEnabledBusiness}\cite{itEnabledBusiness}, tem criado e mantido flexível redes de negócios, inclusive transformando a forma como realizamos negócios e atividades.
	
	Dentro da área de computação observamos a automação agilizar e refinar tarefas como automação de testes, desenvolvimento e manutenção de projetos, devido ao reconhecimento de que o desenvolvimento de software muitas vezes consiste na criação sistemática de componentes que devem aderir a um conjunto bem específico de restrições \cite{automionSoftEvolutionEffect}.
	
	Segundo \citeauthor{automionSoftEvolutionEffect}\cite{automionSoftEvolutionEffect}, a automação do processo de desenvolvimento tem o potencial de reduzir o erro humano em código que deve se adequar a sintaxe e restrições precisas, podendo inclusive produzir software de melhor qualidade que o produzido manualmente, considerando que o talento em desenvolvimento de software é escaço, representando também uma redução de custo. Não o bastante reduz a necessidade de interação humana com tarefas secundárias ou de pouco interesse no desenvolvimento de software, contribuindo para redução da complexidade da tarefa.
	
	Dentre tarefas comuns na Área de TI temos a automatização de tarefas administrativas via linguagens e ferramentas tais como um shell e linguagem script especifica, ou ainda ferramentas de automatização focadas em tarefas de desenvolvimento, como Make, ferramenta de automação de builds , ou como Robot Framework, ferramenta voltada a automação de testes \cite{shell,make,robotFW}.

	Considerando que TI não é a única área onde tarefas repetitivas, bem definidas e restritas ocorrem, o interesse em automatizar estas tarefas resultou em alguns softwares tais como AutoIt para plataforma Windows, um programa que permite automatizar programas com interface por meio da descrição da automatização em uma linguagem script, podendo gerar executáveis independentes que rodam em computadores que não tenham o o autoit instalado, tendo a disposição um grande diversidade de bibliotecas com funções prontas, tendo inclusive uma IDE voltada para sua linguagem de automação. ou ainda o Automator que permite automatizar tarefas repetitivas em plataformas Macintosh que permite construir workflows por meio de unidades modulares chamadas ações; apesar de conter diversas ações pre estabelecidas, é possível inserir novas ações por meio de linguagens como AppleScript e Objective-C \cite{autoit,automator}.
	

	\subsection{Programação visual}
	
	Segundo \citeauthor{visualProgram}\cite{visualProgram} programação visual é "o uso de representações gráficas significativas no processo de programação" realizada em uma linguagem de programação visual que \citeauthor{visualProgram} define como "uma linguagem que utiliza alguma representação visual para completar o que outrora deveria ser escrito em uma linguagem uni-dimensional tradicional". Esta definição hoje tem se mostrado um tanto ampla e tem apresentado diversas contextualizações como podemos ver em \cite{visualProgAuth} que apresenta programação visual no contexto de autoria multimídia.
	
	No contexto deste projeto, programação visual representa a a programação de tarefas de automatização de programas com interface gráfica, com base em elementos visuais por meio do blockly uma biblioteca open source destinada a criação de editores para programação visual, totalmente baseada em tecnologias web e extremamente portável já que além disso é uma ferramenta 100\% client-side, e também extremamente leve funcionando na maioria dos navegadores web e em dispositivos móveis \cite{Blockly}.
	
	Pode ser integrado a qualquer aplicação cuja a linguagem ofereça um web widget, e é capaz de oferecer teste de unidade, possibilidade de tradução e inclusive tratamento de eventos a construção de blocos customizáveis. O processo de criação dos consiste em definir o bloco definindo seu formato, campos e pontos de conexão, o que pode ser realizado com o uso do Block Factory ou ainda da API JSON. A definição de blocos permite inclusive mutabilidade. Em seguida é criado o gerador de código para que o novo bloco possa ser exportado para alguma linguagem de programação.
	
	~\\
	Um exemplo tipico de definição de um bloco:
	

	\begin{lstlisting}[frame=single]
Blockly.Blocks['text_length'] = {
  init: function() {
	this.setColour(160);
	this.appendValueInput('VALUE')
		.setCheck('String')
		.appendField('length of');
	this.setOutput(true, 'Number');
	this.setTooltip('Returns number of 
		letters in the provided text.');
	this.setHelpUrl('http://www.w3schools.com/
		jsref/jsref_length_string.asp');
  }
};
    \end{lstlisting}
    
	~\\
	É voltada ao desenvolvimento de programas embora não seja escalável para grandes programas, pode ainda ser usado como um editor de linguagens visuais para áreas especificas, apresentando elementos comuns a linguagens de programação tais como funções, variáveis, arrays, checagem básica de tipos e afins. Por ser uma ferramenta para criação de editores de linguagens visual, impossibilita que o usuário caia em erros de sintaxe ao fazer programas via Blockly, oferecendo suporte a diversos idiomas e a extensão por via de blocos customizados, possibilitando que os programas sejam exportados para linguagens convencionais.

	A flexibilidade da ferramenta da ferramenta pode ser observada na utilização para criação de jogos, apps móveis para android, programação web ou ainda como recurso educacional \cite{blocklyGames,blocklymobile,blocklyJavaScript,blocklyEducation}.
	
	\subsection{Modelos matemáticos de simulação de culturas agrícolas}
	
	No projeto serão usados modelos matemáticos que simulam diversos processos eco fisiológicos de culturas Agrícolas. Em geral estes modelos trabalham dentro de um ciclo diários, em função de n variáveis tais como, dados meteorológicos que englobam temperatura, umidade do ar, radiação solar, precipitação, nível de CO2 atmosférico, dentre outros com base na localização geográfica a ser simulada, alguns modelos também utilizam a condições do solo que podem envolver balanço hídrico e nível de nutrientes disponível e por fim, o tipo de cultivar sendo simulada, no caso como variáveis que descrevem como ocorre seu desenvolvimento \cite{simanihotArt}. Dentre os modelos que serão usados para estudo de caso em conjunto com a ferramenta desenvolvida no projeto, estão o Simanihot, SoySim e DSSAT.

	O Simanihot é um modelo matemático dinâmico baseado em processos (process – based model), foi projetado para trabalhar em siclos diários e simula diversos processos eco fisiológicos da cultura da Mandioca. O modelo foi desenvolvido pelo grupo Agrometeorológico da Universidade Federal de Santa Maria e é destinado a simular o crescimento, desenvolvimento e produtividade da cultura em questão no estado do Rio Grande do Sul, Brasil. O modelo utiliza como dados de entrada, a datas de plantio e de colheita, dados meteorológicos e balanço hídrico do solo sendo um programa disponibilizado de forma gratuita \cite{Simanihot}.

	Outro modelo usado neste projeto, o SoySim, simula o desenvolvimento da cultura de soja e foi desenvolvido pela Universi de Nebraska-Lincoln. Este modelo simula o potencial de rendimento assim como as necessidades de irrigação, sem limitação pela irrigação e assumindo suplemento ideal de nutrientes e sem perdas de rendimento por influências do ecossistema, tais como granizo, ou de outros meios, tais como envenenamento por nitritos ou nitratos, o siclo de simulação deste modelo também é diário \cite{SoySim}.
	
	Já o DSSAT, por sua vez é um programa que engloba diversos modelos de simulação de cultura agrícola, num total de 42 culturas. Oferece suporte ao manejamento de solo, clima e culturas assim como dados experimentais, por via também de utilitários e outros programas. Os modelos disponíveis simulam o crescimento, desenvolvimento e potencial como uma função de condições agrometeorológicas, tendo sido usado tanto no refino de manejo, em fazendas ou para analise te impacto climático sobre as culturas suportadas \cite{dssat}.
	
	
	\section{Metodologia}
	
	Este estudo caracteriza-se como uma pesquisa exploratória com realização de um estudo de campo de um grupo de pessoas da área de agrometeorologia no intuito de automatizar pesquisas no campo de climas futuros com simulações nos modelos matemáticos SoySim, Simanihot e DSSAT. Se utilizando de uma abordagem qualitativa.
	
	O Estudo sera realizado com o auxilio de pessoas interessadas na automatização de tarefas no prédio de fitotecnia na Universidade Federal de Santa Maria.

	\subsection{Análise de Requisitos}
	
	 Ponto onde serão elencados os requisitos iniciais para a desenvolver o projeto, neste ponto vão ser elencados os casos de uso de automação de tarefas nas interfaces gráficas dos modelos sugeridos, consistindo em elencar elementos automatizáveis, abordagens de automação e conexão destes elementos automatizáveis em tarefas concretas e meios de interação com a abordagem de automatização, requisitos estes descritos em casos de teste.
	
	\subsection{Especificação}
	Fase em que os requisitos serão utilizados para gerar a especificação inicial do projeto a ser desenvolvido com base nas ferramentas de software que serão usadas durante o projeto, a linguagem de programação Java e a biblioteca para construção de editores de programação visual Blockly.
	
	
	\subsection{Implementação, Teste e Revisão}
	
	Desenvolvimento que será realizado em milestones semanais com teste de unidade constantes assim como constante reintegração e revisão dos requisitos. Com a participação de usuários no processo de teste e revisão de requisitos     
    
    
    \subsection{Verificação e Validação}
    
    Integração do usuário ao software de automação desenvolvido com o objetivo de avaliar se a ferramenta é capaz de suprir as necessidades de automação de forma intuitiva, analisando quais as resistências do usuário a metodologia de automação, observando as dificuldades envolvendo os domínios necessários para a ferramenta de automação. Assim como analise da contribuição da ferramenta as atividades do usuário. 

	\section{Plano de Atividades e Cronograma}
	
	O cronograma de atividades será composto de 5 etapas, são elas:
	
	\begin{enumerate}
		\item \label{activity:requisitos} \textbf{Recolhimento de requisitos e estudo da abordagem na linguagem visual \emph{Blockly}}
		Nesta etapa serão recolhidos requisitos suficientes para a proposta inicial do projeto de automatizar tarefas em modelos de simulação de culturas agrícolas, mais especificamente nos simuladores definidos, seguindo para a especificação da abordagem de desenvolvimento do programa e de uma linguagem visual adequada ao projeto com base nas ferramentas definidas e requisitos coletados.
		\item \label{activity:dev1} \textbf{Desenvolvimento do software base:}
		Momento em que será desenvolvida a interface gráfica do aplicativo, como sera realizada a detecção de outros programas e como o software ira interagir com estes programas. 
		\item \label{activity:dev2} \textbf{Desenvolvimento de uma API de automatização visual:}
		etapa que em que será desenvolvida a abordagem de automação dos programas, como sera tratada a contextualização, separação entre elementos e tarefas e como estes serão conectados.
		\item \label{activity:revisão} \textbf{Revisão e teste:}
		Etapa constante de revisão dos requisitos e especificação do projeto, assim como implementação e realização de testes dos componentes do software em desenvolvimento.
		\item \label{activity:avaliação} \textbf{Avaliação do software em uso:}
		Avaliação da competência do software em automatizar os modelos de simulação de culturas agrícolas sugeridos, em uso no centro de Fitotecnia da Universidade Federal de Santa Maria, avaliando a resistência do usuário a ferramenta, dependência de treinamento e efetividade na automatização das tarefas dos modelos.
	\end{enumerate}
	
	\begin{table}[ht]
		\centering
		\begin{tabular}{c|ccccc}
			Etapa & Março & Abril & Maio & Junho & Julho \\ \hline
			\ref{activity:requisitos} & \checkmark & & & & \\
			\ref{activity:dev1} & \checkmark & \checkmark & \checkmark & & \\
			\ref{activity:dev2} & & \checkmark & \checkmark & \checkmark & \\
			\ref{activity:revisão} & \checkmark & \checkmark & \checkmark & \checkmark & \checkmark \\
			\ref{activity:avaliação} & & & & \checkmark & \checkmark \\
		\end{tabular}
		\caption{Cronograma de Atividades}
	\end{table}
	
	
	\section{Recursos}
	
	Como recursos físicos serão utilizados nesse trabalho, o computador pessoal do pesquisador e linguagens de programação e bibliotecas open source para a produção do software, Também serão utilizados os programas definidos neste projeto da área agronômica para modelagem matemática, como ambiente de teste para a proposta do projeto.

	O projeto envolvera como usuários da ferramenta estudantes da UFSM na área de Agronomia, do Centro de Fitotecnia da UFSM. 

	\section{Resultados Esperados}
	
	É esperado ao fim do projeto, automatizar tarefas em modelos de simulação de culturas agrícolas sem a necessidade direta de uma linguagem textual formal mas sim por meio da expressão visual proporcionada por uma linguagem de programação visual contextualizada dentro das necessidades dos usuários destes modelos. 

	\bibliographystyle{abbrvnat}
	\bibliography{projeto} 
	
\end{document}